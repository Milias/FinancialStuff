\documentclass[8 pt]{article}

\usepackage[utf8x]{inputenc}
\usepackage{dsfont}
\usepackage{amsthm}
\usepackage{amsfonts}
\usepackage{amssymb}
\usepackage{tensor}
\usepackage{mathtools}
\usepackage[T1]{fontenc}
%\usepackage[spanish]{babel}
\usepackage[cm]{fullpage}
\usepackage{graphicx}
%\usepackage{float}
\usepackage{bm}
\usepackage{setspace}
\usepackage{enumitem}
\usepackage{mdwlist}
\usepackage{parskip}
\usepackage{listings}
\usepackage{color}
%\usepackage{epstopdf}
\usepackage{tikz,datatool}
\usepackage{hyperref}
\usepackage{mathabx}
\usepackage{multicol}
\usepackage{eurosym}
\usepackage{caption}

\newcommand{\HRule}{\rule{\linewidth}{0.5mm}}

\AtBeginDocument{
  \let\myThePage\thepage
  \renewcommand{\thepage}{\oldstylenums{\myThePage}}
}

\newcommand{\gra}{$^\text{o}$}
\newcommand{\dif}{\text{d}}
\newcommand{\avg}[1]{\left\langle #1 \right\rangle}
\newcommand{\ket}[1]{\left| #1 \right\rangle}
\newcommand{\bra}[1]{\left\langle #1 \right|}
\newcommand{\bket}[2]{\left\langle #1 \middle| #2 \right\rangle}
\newcommand{\der}[2]{\frac{\text{d} #1}{\text{d} #2}}
\newcommand{\prt}[2]{\frac{\partial #1}{\partial #2}}
\newcommand{\dert}[3]{\frac{\text{d}^#3 #1}{\text{d} #2^#3}}
\newcommand{\prtt}[3]{\frac{\partial^#3 #1}{\partial #2^#3}}
\newcommand{\dl}{\mathcal{L}}
\newcommand{\dha}{\mathcal{H}}
\newcommand{\vol}{\text{vol}}
\renewcommand{\vec}[1]{\pmb{#1}}

\DeclarePairedDelimiter\ceil{\lceil}{\rceil}
\DeclarePairedDelimiter\floor{\lfloor}{\rfloor}

\newenvironment{Figure}
  {\par\medskip\noindent\minipage{\linewidth}}
  {\endminipage\par\medskip}

\begin{document}

\begin{minipage}{\textwidth}
    \centering
    \Large \textbf{\textsc{Homework 3: Dual Listing Arbitrage}}
    \vspace{0.5cm}

    \small \textsc{Francisco García Flórez, Joris van Lammeren, Wouter Varenkamp}
    \vspace{0.5cm}

    \begin{minipage}{0.6\textwidth}
      \textbf{Abstract.} In this homework we study pure arbitrage as a way of making risk-free profits by trading some volume between two markets. Even though the profits per trade are low compared to the amount of cash involved, it is still possible to make a significant profit over time.
    \end{minipage}
\end{minipage}

\vspace{0.5cm}

\begin{multicols*}{2}
\section{Conclusion}

<<<<<<< HEAD
\section{Trading robot}

\subsection{Algorithm}

\subsection{Implementation}

\section{Results}

\begin{center}
  \begin{tabular}{|c|c|c|c|c|c|}
    \hline
    \# & CHI (Buy) & CHI (Sell) & EUR (Buy) & EUR (Sell) & Profit \\ \hline
    1 & -37407.95 & 74623.63 & -74613.81 & 37412.86 &     14.73 \\ \hline
2 & -57272.87 & 24620.31 & -24617.09 & 57280.36 &     10.71 \\ \hline
3 & -37686.14 & 29292.16 & -29288.31 & 37691.10 &      8.81 \\ \hline
4 & -19650.57 & 85786.14 & -85774.92 & 19653.14 &     13.79 \\ \hline
5 & -66249.23 & 57756.53 & -57748.97 & 66259.82 &     18.15 \\ \hline

  \end{tabular}
\end{center}

\begin{center}
  \begin{tabular}{|c|c|c|c|c|c|}
    \hline
    \# & CHI (Buy) & CHI (Sell) & EUR (Buy) & EUR (Sell) & Pos. \\ \hline
    1 &       491 &     -982 &       982 &     -491 &         0 \\ \hline
2 &       749 &     -322 &       322 &     -749 &         0 \\ \hline
3 &       496 &     -385 &       385 &     -496 &         0 \\ \hline
4 &       257 &    -1122 &      1122 &     -257 &         0 \\ \hline
5 &       866 &     -756 &       756 &     -866 &         0 \\ \hline

  \end{tabular}
\end{center}

Our main conclusion is that it takes a high investment to make a small profit. This seems like it is not a good investment,
but there is no chance on making a loss. So we make a small risk-free profit without the risk of making a loss.

\begin{thebibliography}{28}
\raggedright
\addcontentsline{toc}{section}{Bibliography}

\bibitem{Wilmott} P. Wilmott et al, \emph{The Mathematics of Financial Derivatives}, 1995.

\end{thebibliography}

\end{multicols*}

\end{document}
